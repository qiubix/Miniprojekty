\documentclass[12pt,a4paper,notitlepage]{article}
%\usepackage{polski}
\usepackage[T1]{fontenc}
\usepackage[polish]{babel}
\usepackage[utf8]{inputenc}
%\usepackage[cp1250]{inputenc}
\usepackage{lmodern}
\selectlanguage{polish}
\usepackage[top=3cm, bottom=2cm, left=1.5cm, right=1.5cm]{geometry}
\usepackage{fancyhdr}
\usepackage{graphicx}

% nazwy kolorow
\usepackage{color}
% listingi
\usepackage{listings}
\definecolor{dkgreen}{rgb}{0,0.39063,0}
\definecolor{gray}{rgb}{0.5,0.5,0.50}
\definecolor{mauve}{rgb}{0.77,0.77,0.77}
\definecolor{purple}{rgb}{0.6,0,0.5}
\lstset{ %
  language=Matlab,                % the language of the code
  basicstyle=\footnotesize,           % the size of the fonts that are used for the code
  numbers=left,                   % where to put the line-numbers
  numberstyle=\tiny\color{gray},  % the style that is used for the line-numbers
                                  % will be numbered
  numbersep=5pt,                  % how far the line-numbers are from the code
  backgroundcolor=\color{mauve},      % choose the background color. You must add \usepackage{color}
  showspaces=false,               % show spaces adding particular underscores
  frame=single,                   % adds a frame around the code
  rulecolor=\color{black},        % if not set, the frame-color may be changed on line-breaks within not-black text (e.g. comments (green here))
  tabsize=2,                      % sets default tabsize to 2 spaces
  captionpos=b,                   % sets the caption-position to bottom
  breaklines=true,                % sets automatic line breaking
                                  % also try caption instead of title
  keywordstyle=\color{blue},          % keyword style
  commentstyle=\color{dkgreen},       % comment style
  stringstyle=\color{purple},         % string literal style
}



\begin{document}
\lhead{Karol Katerżawa}
\chead{Laboratorium 3. i 4.}
\rhead{}

\pagestyle{fancy}

\section{Opis metody klasyfikacji znaków przy użyciu klasyfikatorów liniowych}
Przy wczytywaniu danych wykorzystałem funkcję \textit{readSets()} z rozwiązania referencyjnego. Do redukcji wymiarowości algorytmem PCA także zastosowałem funkcje z rozwiązania referencyjnego. Dane wejściowe przekształciłem do 40-wymiarowej przestrzeni PCA. 
\subsection{Opis algorytmu}
\begin{enumerate}
\item Wykorzystując metodę perceptronową wyznaczam parametry płaszczyzn decyzyjnych dla 45 klasyfikatorów liniowych \textit{one-vs-one}
\item Pobieram próbkę ze zbioru testowego
\item Kolejne klasyfikatory głosują na jedną z dwóch cyfr, które rozpoznają. 
\item Jeśli więcej niż jedna cyfra otrzymała maksymalną liczbę głosów, daj decyzję wymijającą. 
\item Wybierz dla próbki etykietę, która otrzymała maksymalną liczbe głosów. 
\item Jeśli są jeszcze próbki w zbiorze testowym, wróć do punktu 2. 
\end{enumerate}


\section{Algorytm wyznaczania parametrów płaszczyzny decyzyjnej}
\begin{enumerate}
\item Dodaj wektor jedynek na początku macierzy z cechami pierwszej klasy.
\item Dodaj wektor minus-jedynek na początku macierzy z cechami drugiej klasy. 
\item Wylosuj parametry płaszczyzny sterującej ( przedział (-0.5;0.5) ) i wpisz je do wektora res.
\item Przyjmij $\mu = 1$.
\item Dla każdej próbki wylicz $ J = w * A' $, gdzie $w$ jest wektorem cech danej próbki. 
\item Wylicz $ \nabla J $ = suma wartości cech próbek z $ J < 0$.
\item Wylicz $res(k+1)=res(k) + \frac{\nabla J  }{\sqrt{\mu}}$.
\item Inkrementuj $\mu$.
\item Jeśli $\nabla J > 0.01 $ oraz $\mu < 1000 $ wróć do punktu 5. 
\end{enumerate}

\section{Dane wynikowe}

\subsection{Jakość poszczególnych klasyfikatorów \textit{one-vs-one}}
\begin{center}
\begin{tabular}{|c|c|c|c|}
\hline
   0 & 1 &  0.99811 & 0.00189\\\hline
   0 & 2 &  0.98708 & 0.01292\\\hline
   0 & 3 &  0.99548 & 0.00452\\\hline
   0 & 4 &  0.99796 & 0.00204\\\hline
   0 & 5 &  0.98558 & 0.01442\\\hline
   0 & 6 &  0.98555 & 0.01445\\\hline
   0 & 7 &  0.99502 & 0.00498\\\hline
   0 & 8 &  0.99079 & 0.00921\\\hline
   0 & 9 &  0.99246 & 0.00754\\\hline
   1 & 2 &  0.99123 & 0.00877\\\hline
   1 & 3 &  0.99114 & 0.00886\\\hline
   1 & 4 &  0.99811 & 0.00189\\\hline
   1 & 5 &  0.99161 & 0.00839\\\hline
   1 & 6 &  0.99379 & 0.00621\\\hline
   1 & 7 &  0.99399 & 0.00601\\\hline
   1 & 8 &  0.98720 & 0.01280\\\hline
   1 & 9 &  0.99487 & 0.00513\\\hline
   2 & 3 &  0.97502 & 0.02498\\\hline
   2 & 4 &  0.98213 & 0.01787\\\hline
   2 & 5 &  0.97401 & 0.02599\\\hline
   2 & 6 &  0.97688 & 0.02312\\\hline
   2 & 7 &  0.97427 & 0.02573\\\hline
   2 & 8 &  0.97657 & 0.02343\\\hline
   2 & 9 &  0.97991 & 0.02009\\\hline
   3 & 4 &  0.99398 & 0.00602\\\hline
   3 & 5 &  0.94585 & 0.05415\\\hline
   3 & 6 &  0.99390 & 0.00610\\\hline
   3 & 7 &  0.97792 & 0.02208\\\hline
   3 & 8 &  0.96119 & 0.03881\\\hline
   3 & 9 &  0.97821 & 0.02179\\\hline
   4 & 5 &  0.98933 & 0.01067\\\hline
   4 & 6 &  0.98814 & 0.01186\\\hline
   4 & 7 &  0.98408 & 0.01592\\\hline
   4 & 8 &  0.99080 & 0.00920\\\hline
   4 & 9 &  0.95932 & 0.04068\\\hline
   5 & 6 &  0.97027 & 0.02973\\\hline
   5 & 7 &  0.99427 & 0.00573\\\hline
   5 & 8 &  0.95284 & 0.04716\\\hline
   5 & 9 &  0.98106 & 0.01894\\\hline
   6 & 7 &  0.99144 & 0.00856\\\hline
   6 & 8 &  0.98758 & 0.01242\\\hline
   6 & 9 &  0.99441 & 0.00559\\\hline
   7 & 8 &  0.98352 & 0.01648\\\hline
   7 & 9 &  0.95385 & 0.04615\\\hline
   8 & 9 &  0.97378 & 0.02622\\\hline
\end{tabular}
\end{center}

\subsection{Klasyfikacja cyfr}
\begin{center}
\textbf{Macierz pomyłek:}\\
\begin{tabular}{|c|c|c|c|}
\hline
Cyfra & True Positive & False Positive & False Negative\\\hline
      0  &  954  &   45  &   38\\\hline
      1  & 1115  &   32  &   29\\\hline
      2  &  955  &  104  &  101\\\hline
      3  &  931  &  126  &  108\\\hline
      4  &  923  &   90  &   70\\\hline
      5  &  782  &   97  &  135\\\hline
      6  &  893  &   45  &   82\\\hline
      7  &  947  &   69  &  106\\\hline
      8  &  867  &   84  &  135\\\hline
      9  &  880  &   61  &  153\\\hline
\end{tabular}
\end{center}

\begin{center}
\begin{tabular}{|c|c|c|c|c|c|c|c|c|c|c|}
\hline
	& \multicolumn{10}{|c|}{...pomylona z:}\\\hline
 Cyfra &  0  &  1 &   2 &   3 &   4 &   5  &  6 &   7  &  8  &  9\\\hline
    0  &  -  &  0 &   4 &   2 &   0 &   7  & 10 &   2  &  1  &  0\\\hline
    1  &  0  &  - &   4 &   2 &   0 &   3  &  1 &   1  &  9  &  0\\\hline
    2  &  8  &  6 &   - &  12 &   7 &   3  & 13 &  12  & 15  &  1\\\hline
    3  &  3  &  2 &  18 &   - &   4 &  28  &  0 &   9  & 11  &  4\\\hline
    4  &  1  &  1 &   6 &   2 &   - &   2  &  6 &   6  &  6  & 29\\\hline
    5  &  9  &  4 &   9 &  51 &   5 &   -  &  7 &   2  & 20  &  3\\\hline
    6  & 10  &  4 &  19 &   3 &   8 &  18  &  - &   1  &  2  &  0\\\hline
    7  &  1  &  4 &  28 &  11 &  13 &   0  &  0 &   -  &  6  & 18\\\hline
    8  &  6  &  5 &  11 &  30 &   7 &  29  &  7 &   6  &  -  &  6\\\hline
    9  &  7  &  6 &   5 &  13 &  46 &   7  &  1 &  30  & 14  &  -\\\hline
\end{tabular}
\end{center}

\begin{center}
\begin{tabular}{|c|c|c|c|}
\hline
Cyfra & Poprawne & Błąd & Dec. wymijajaca\\\hline
   0 &  0.96122 &  0.02653 &  0.01224\\\hline
   1 &  0.97445 &  0.01762 &  0.00793\\\hline
   2 &  0.90213 &  0.07461 &  0.02326\\\hline
   3 &  0.89307 &  0.07822 &  0.02871\\\hline
   4 &  0.92872 &  0.06008 &  0.01120\\\hline
   5 &  0.84865 &  0.12332 &  0.02803\\\hline
   6 &  0.91441 &  0.06785 &  0.01775\\\hline
   7 &  0.89689 &  0.07879 &  0.02432\\\hline
   8 &  0.86140 &  0.10986 &  0.02875\\\hline
   9 &  0.84836 &  0.12785 &  0.02379\\\hline
   Razem &  0.90293 &  0.07647 &  0.02060\\\hline
\end{tabular}
\end{center}



\subsection{Wnioski}
\subsubsection{Jakość klasyfikatorów}
\begin{center}
\begin{tabular}{|c|c|}\hline
Błąd & Wartość \\\hline
Średni & 0.016567	\\\hline
Maksymalny & 0.054154	 \\\hline
Minimalny & 0.0018895	\\\hline
\end{tabular}
\end{center}

\subsubsection{Klasyfikacja cyfr}
Najlepszy wynik uzyskano przy rozpoznaniu cyfr 0, 1 i 4 mimo, iż klasyfikatory 0vs5, 1vs8 i 4vs9 mają bardzo słabą dokładność. Jednakże te cyfry mają stosunkowo niski wynik FP w macierzy pomyłek, 4 jest mylona z cyfrą 9 aż 29 razy, ale za to z innymi cyframi bardzo rzadko. Obserwując macierz pomyłek łatwo można zauważyć, że duże trudności przy klasyfikacji sprawia cyfra 5, która oprócz dużego błędu ma też duży procent decyzji wymijających. Dla porównania cyfra 9 ma również wysoki współczynnik błędu, gdyż jest często mylona z cyfrą 4 i 7, jednak decyzje wymijające są podejmowane stosunkowo rzadko. 


\section{Niestandardowy sposób podejmowania decyzji}
Ponieważ celem jest zmniejszenie liczby błędów przy jak najmniejszej stracie na liczbie decyzji poprawnych sensownym rozwiązaniem wydaje się dodanie do głosowania \textit{one-vs-one} dodatkowego głosu \textit{one-vs-rest}. Sam klasyfikator \textit{one-vs-rest} dla takiego zadania jest mało użyteczny, gdyż będzie dawał zbyt dużo decyzji wymijających, dlatego według mnie powinien służyć jako klasyfikator pomocniczy oprócz tych 45 klasyfikatorów \textit{one-vs-one}. Dzięki temu na każdą klasę będzie potencjalnie jeden głos więcej, co zmniejszy liczbę błędów. 

\begin{center}
\textbf{Macierz pomyłek:}\\
\begin{tabular}{|c|c|c|c|}
\hline
Cyfra & True Positive & False Positive & False Negative\\\hline
      0  &  960  &   55  &   36\\\hline
      1  & 1116  &   34  &   38\\\hline
      2  &  958  &  109  &  114\\\hline
      3  &  920  &  144  &  131\\\hline
      4  &  931  &  101  &   84\\\hline
      5  &  759  &   94  &  187\\\hline
      6  &  882  &   36  &  103\\\hline
      7  &  946  &   71  &  105\\\hline
      8  &  848  &  106  &  177\\\hline
      9  &  868  &   62  &  175\\\hline
\end{tabular}
\end{center}


\begin{center}
\begin{tabular}{|c|c|c|c|}
\hline
Cyfra & Poprawne & Błąd & Dec. wymijajaca\\\hline
   0	&  0.96327  & 0.02041  &  0.01633\\\hline
   1	&  0.96652  & 0.01674  &  0.01674\\\hline
   2	&  0.88953  & 0.07171  &  0.03876\\\hline
   3	&  0.87030  & 0.08911  &  0.04059\\\hline
   4	&  0.91446  & 0.05193  &  0.03360\\\hline
   5	&  0.79036  & 0.14910  &  0.06054\\\hline
   6	&  0.89248  & 0.07933  &  0.02818\\\hline
   7	&  0.89786  & 0.07977  &  0.02237\\\hline
   8	&  0.81828  & 0.12936  &  0.05236\\\hline
   9	&  0.82656  & 0.13974  &  0.03370\\\hline
   Razem&  0.88296  & 0.08272  &  0.03432\\\hline 
\end{tabular}
\end{center}

\begin{center}
\begin{tabular}{|c|c|c|c|}
\hline
POPR & 0.98950  & 0.00144 & 0.00906\\\hline
BŁ & 0.01752  & 0.98248 & 0.00000\\\hline
WYM & 0.32534  & 0.04452 & 0.71918\\\hline  
\end{tabular}
\end{center}

\end{document}



