\documentclass[12pt,a4paper,notitlepage]{article}
%\usepackage{polski}
\usepackage[T1]{fontenc}
\usepackage[polish]{babel}
\usepackage[utf8]{inputenc}
\usepackage{lmodern}
\selectlanguage{polish}
\usepackage[top=3cm, bottom=2cm, left=2cm, right=2cm]{geometry}
\usepackage{fancyhdr}

% nazwy kolorow
\usepackage{color}
% listingi
\usepackage{listings}
\definecolor{dkgreen}{rgb}{0,0.39063,0}
\definecolor{gray}{rgb}{0.5,0.5,0.50}
\definecolor{mauve}{rgb}{0.77,0.77,0.77}
\lstset{ %
  language=Matlab,                % the language of the code
  basicstyle=\footnotesize,           % the size of the fonts that are used for the code
  numbers=left,                   % where to put the line-numbers
  numberstyle=\tiny\color{gray},  % the style that is used for the line-numbers
                                  % will be numbered
  numbersep=5pt,                  % how far the line-numbers are from the code
  backgroundcolor=\color{white},      % choose the background color. You must add \usepackage{color}
  showspaces=false,               % show spaces adding particular underscores
  frame=single,                   % adds a frame around the code
  rulecolor=\color{black},        % if not set, the frame-color may be changed on line-breaks within not-black text (e.g. comments (green here))
  tabsize=2,                      % sets default tabsize to 2 spaces
  captionpos=b,                   % sets the caption-position to bottom
  breaklines=true,                % sets automatic line breaking
                                  % also try caption instead of title
  keywordstyle=\color{blue},          % keyword style
  commentstyle=\color{dkgreen},       % comment style
  stringstyle=\color{mauve},         % string literal style
}



\begin{document}
\lhead{Karol Katerżawa}
\chead{}
\rhead{ROB - Ćwiczenie 2.}

\pagestyle{fancy}
\section*{Klasyfikacja optymalna Bayesa}
\subsection*{Sprawdzenie danych}
Obliczenie skrajnych wartości zbioru testowego, analiza danych statystycznych a także obejrzenie histogramów spowodowało odrzucenie ze zbioru testowego próbek numer \textbf{186} oraz \textbf{641}.
\subsection*{Wybór cech}
Przy pomocy programu \textit{plot2features} wyświetliłem rozkłady różnych par cech i wybrałem ostatecznie cechy \textbf{drugą} i \textbf{trzecią}.

\subsection*{Budowa klasyfikatorów}
Klasyfikatory ponumerowałem zgodnie z numeracją metod w instrukcji. Na początku przyjąłem $h_1 = 0,01$ (oczywiście jest ono uwzględniane tylko w klasyfikatorze korzystającym z okna Parzena).

\begin{center}
\begin{tabular}{|c|c|}
\hline
Nr. klasyfikatora	&	Jakość klasyfikatora \\\hline
	1				&		0.84978		\\\hline
	2				&		0.97697		\\\hline
	3				&		0.61568		\\\hline
\end{tabular}
\end{center}

\subsection*{Dobór szerokości okna}
Od razu widać, że szerokość okna Parzena jest bardzo źle dobrana, bo jakość klasyfikatora wyszła bardzo słaba, a wręcz niebezpiecznie zbliżyła się do 50\%. Tabelka przedstawia zmianę jakości klasyfikatora przy zmianie szerokości okna. 
\begin{center}
\begin{tabular}{|c|c|}
\hline
	$h_1$	&	Jakość klasyfikatora \\\hline
	0.01		&		0.61568		\\\hline
	0.001		&		0.74945		\\\hline
	0.0001		&		0.98958		\\\hline
	0.00001 	&		0.99561		\\\hline
	0.000001	&		0.98958		\\\hline
	0.0000001	&		0.71107		\\\hline	
\end{tabular}
\end{center}
Wyraźnie widać, że przy zmniejszaniu szerokości okna jakość klasyfikatora się poprawia do momentu osiągnięcia wartości optymalnej. Dalsze zmniejszanie okna powoduje znów pogarszanie jakości klasyfikatora. 

\subsection*{Różne zbiory testowe}
Ocena wpływu doboru różnej wielkości zbiorów uczącychc na klasyfikację zbioru testowego. 
\begin{center}
\begin{tabular}{|c|c|c|c|}
\hline
Część zbioru train	&	Klasyfikator 1 	&	Klasyfikator 2 	&	Klasyfikator 3 \\\hline
	$\frac{1}{10}$	&		0.84978		&					&					\\\hline
	$\frac{1}{4}$	&		0.97697		&					&					\\\hline
	$\frac{1}{2}$	&		0.61568		&					&					\\\hline
	$1$				&		0.61568		&					&					\\\hline
\end{tabular}
\end{center}

\subsection*{Zmiana prawdopodobieństwa \textit{a priori}}



\subsection*{Porównanie z klasyfikatorem 1-NN}
Uruchomienie klasyfikatora 1-NN na danych kart dało rezultat:
\begin{verbatim}
quality =  0.99561
\end{verbatim}



\end{document}



